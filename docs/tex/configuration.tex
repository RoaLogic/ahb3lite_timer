\chapter{Configuration} \label{configurations}

\section{Introduction}\label{introduction}

The size and implementation style of the timer module is defined via HDL
parameters as specified below.

\section{Core Parameters}\label{core-parameters}

\begin{longtable}[]{@{}lccl@{}}
\toprule
Parameter & Type & Default & Description\tabularnewline
\midrule
\endhead
\texttt{TIMERS}      & Integer & 3  & Number of Timers\tabularnewline
\texttt{HADDR\_SIZE} & Integer & 32 & Width of AHB-Lite Address Bus\tabularnewline
\texttt{HDATA\_SIZE} & Integer & 32 & Width of AHB-Lite Data Buses\tabularnewline
\bottomrule
\end{longtable}

\subsection{TIMERS}\label{timers}

The parameter \texttt{\texttt{TIMERS}} defines the number of timers supported and thereby
the number of \texttt{TIMECMP} registers implemented by the core. Values between
1 and 32 are supported, with the default defined as `3'.

\subsection{HADDR\_SIZE}\label{haddr_size}

The \texttt{HADDR\_SIZE} parameter specifies the address bus size to connect to
the AHB-Lite based host.

\subsection{HDATA\_SIZE}\label{hdata_size}

The \texttt{HDATA\_SIZE} parameter specifies the data bus size to connect to the
AHB-Lite based host. The maximum size supported is 64 bits.

\section{Core Registers}\label{core-registers}

\begin{longtable}[]{@{}lllll@{}}
\toprule
Register & Address & Size & Access & Function\tabularnewline
\midrule
\endhead
\texttt{PRESCALER}  & Base + 0x00    & 32bits & Read/Write & Timebase\tabularnewline
\texttt{IPENDING}   & Base + 0x08    & 32bits & Read Only  & Interrupt Pending\tabularnewline
\texttt{IENABLE}    & Base + 0x0C    & 32bits & Read/Write & Interrupt Enable\tabularnewline
\texttt{TIME}       & Base + 0x10    & 64bits & Read/Write & Timer Register\tabularnewline
\texttt{TIMECMP[n]} & Base + 0x18+8n & 64bits & Read/Write & Compare Value\tabularnewline
\bottomrule
\end{longtable}

Note: `n' represents an integer for 0 to TIMERS-1.

\subsection{PRESCALER}\label{prescaler}

The Timer module operates synchronously with the AHB-Lite bus clock
input \texttt{HCLK}. A 32 bit \texttt{PRESCALER} register enables the time base for the
timers to be less than that of \texttt{HCLK} by dividing this clock frequency by
the value of \texttt{PRESCALER} + 1.

For example: If \texttt{PRESCALER}=3, the timer will increment every \texttt{PRESCALER}+1=4
cycles of HCLK, setting the time base to HCLK/4 Hz.

The default value of \texttt{PRESCALER}=0, thereby setting the timer clock
frequency equal to the bus (\texttt{HCLK}) frequency. The TIME counter starts
incrementing once the register \texttt{PRESCALER} is written to for the first
time.

Note: The value of \texttt{PRESCALER} value can only be defined once after the
peripheral is released from reset.

\subsection{IPENDING}\label{ipending}

\texttt{IPENDING} is a 32-bit read-only register that indicates if a timer
interrupt is pending.

Each bit of the \texttt{IPENDING} register corresponds to one timer with the
position of each bit indicating the associated timer. E.g. bit zero
indicates the interrupt status of \texttt{Timer[0]}. \texttt{IPENDING} bits associated
with unimplemented timers are tied low (`0')

An interrupt pending bit is set when the value of \texttt{TIMECMP[n]} equals
the value of \texttt{TIME}. It is cleared by a write to the associated
\texttt{TIMECMP[n]} register, as specified in the RISC-V privileged spec
1.9.1.

\subsection{IENABLE}\label{ienable}

\texttt{IENABLE} is a 32-bit Read/Write register, where each bit of the register
is a dedicated 'Interrupt Enable' bit for each time. The bit position
indicates the associated timer. E.g. Interrupt Enable for \texttt{Timer[0]}
is located at bit position 0.

Only \texttt{TIMERS} bits are implemented with the remaining MSBs always read as
'0'. A write to the unused MSBs has no effect.

An interrupt is generated when a bit of \texttt{IPENDING} is set and its
associated \texttt{IENABLE} bit is also set. This allows the core to be used in
(1) pure POLL mode, where the CPU polls the status of the bits to
determine if a timer event happened, (2) pure interrupt driven mode,
where each timer can generate an interrupt, or (3) a combination of the
above.

\subsection{TIME}\label{time}

The \texttt{TIME} register is a common 64-bit high-resolution time-keeping
counter used by all timers. It is the basis for the \texttt{RDCYCLE} instruction
as specified in the RISC-V privileged spec 1.9.1 and may be written to
also in accordance with the RISC-V specification.

The time base for the \texttt{TIME} register is derived from the AHB-Lite bus
clock \texttt{HCLK} and is defined as:

Freq\textsubscript{TIME} = Freq\textsubscript{HCLK} / (\texttt{PRESCALER}+1)

The counter starts incrementing once the register \texttt{PRESCALER} is written
to for the first time.

\subsection{TIMECMP[n]}\label{timecmpn}

For each timer (as defined by the parameter \texttt{TIMER}) there is a dedicated
64 bit Time Compare register which defines when the \texttt{IPENDING} bits are
asserted

These registers are denoted as \texttt{TIMECMP[n]}, where `n' is an index
from 0 to \texttt{TIMERS}-1, and are located consecutively in the address space
according to the formula:

Base Address of \texttt{TIMECMP[n] = 0x18 + 8n}

For example, \texttt{TIMECMP[0]} is located at address \texttt{0x18}, \texttt{TIMECMP[1]}
at \texttt{0x20}, \texttt{TIMECMP[5]} at \texttt{0x28} etc.

The \texttt{IPENDING} bit associated with the \texttt{TIMECMP} register is set when the
\texttt{TIMECMP[n]} value equals the value of \texttt{TIME}.

\texttt{IPENDING[n] = (TIMECMP[n] == TIME)}

Writing the \texttt{TIMECMP[n]} register clears bit `n' of the \texttt{IPENDING}
register.

